\documentclass[a4paper,12pt,twocolumn]{article}

\usepackage[utf8]{inputenc}
\usepackage{graphicx}
\usepackage{amsmath}
\usepackage{amssymb}
\usepackage{geometry}
\usepackage{hyperref}
\usepackage{fancyhdr}
\usepackage{svg}
\usepackage{natbib}

\geometry{margin=1in}

\fancypagestyle{plain}{
  \fancyhf{}
  \lhead{Sergei Mironov}
  \rhead{Litrepl}
  \rfoot{\thepage}
}

\pagestyle{plain}

\title{Litrepl: Literate Paper Processor Promoting Transparency More Than Reproducibility}
\author{Sergei Mironov \\
        \texttt{sergei.v.mironov@proton.me}}
\date{\today}

\begin{document}

\maketitle

\begin{abstract}
Litrepl is a Python text processor for recognizing and evaluation code sections
in Markdown or LaTeX documents. Litrepl is designed as a middleware aimed at
separating text editors from the programming language interpreter management
logic thus contributing to the diversity of both. In this role, Litrepl can
become a component of modular "UNIX Way" interactive development or typesetting
environments.
\end{abstract}

\section{Statement of Need}

The concept of Literate Programming, initially proposed by Donald Knuth,
revolves around explaining to human beings what we want a computer to do. This
approach is embodied in tools like WEB, CWEB, and NOWEB, which focus on
non-interactive, ahead-of-time compilation. The process includes two primary
modes of operation: weaving, which creates human-readable documentation, and
tangling, which generates machine-executable code. Over time, the concept has
evolved, showing a trend towards simplification \cite{Knuth1984lp,
Ramsey1994lps}.

Concurrently, another concept of human-computer interaction called the
Read-Evaluate-Print Loop (REPL) gained traction, notably within the LISP
community and through the APL language designed for mathematical computations.
The combination of a command-line interface and a language interpreter enables
incremental and interactive programming, allowing users to directly modify the
interpreter state. By maintaining human involvement in the loop, this approach
facilitates human thought processes \cite{McCarthy1959recfun, Iverson1962apl}.

A significant milestone in this area was the development of the IPython project,
which later evolved into the Jupyter Project. The creators introduced a new
document format called the Notebook, characterized by a series of logical
sections of various types, including text and code, which could directly
interact with programming language interpreters. This interactive communication,
akin to REPL style programming, allows the creation of well-structured documents
suitable for presentations and sharing. The concept underpinning these
developments is termed Literate Computing, which includes goals of spanning a
wide audience range, boosting reproducibility, and fostering collaboration. To
achieve these objectives, several technical decisions were made, notably the
introduction of bidirectional communication—between the computational core,
known as the Jupyter Kernel, and the Notebook acting as a client, along with
another layer of client-server communication between the Notebook web server and
the user’s web browser \cite{Perez2007IPython, Granger2021litcomp,
Kluyver2016jupnb}.

4. While we recognize the importance of various goals within the Literate
Computing framework, we argue that reproducibility is paramount. Successfully
addressing reproducibility would alone suffice to enhance communication over
time among research project participants and significantly expand the audience.
However, as it became clear (\cite{Dolstra2010, Vallet2022}), this challenge
extends far beyond the scope of a human-computer interaction system, and even
beyond the typical boundaries of software distribution management for a
particular programming language. A comprehensive solution to the software
deployment problem should operate at the entire operating system level.


5. Consequently, we propose focusing on transparency in human-computer
interaction rather than on reproducibility. We present "Litrepl," a tool
designed to integrate the REPL interactive programming style into existing
editors in a straightforward yet potent manner.

6. First, we utilize simple bidirectional text streams for inter-process
communication with language interpreters to evaluate code. Second, we advocate
for the reuse of existing text document formats. In both the Markdown and LaTeX
formats that we have implemented, simplified parsers are utilized to distinguish
code and result sections from the rest of the document. As of now, we support
Python and Shell interpreter families, as well as a custom AI communication
interpreter. Finally, we strive to leverage operating system facilities as much
as possible.


\section{How it works}


\begin{figure*}[t]
  \centering
  \includesvg[inkscapelatex=false,width=0.8\textwidth]{pic}
  \caption{Caption describing the image shown in the figure.}
  \label{fig:example}
\end{figure*}


\subsection{Inter-process communication}

When the file appears on the Litrepl's stdin stream, it uses Earley parser to
reliably recognize code and result sections.

\subsection{Parsing and Evaluation}

Details and discussions about the second subsection.

\subsection{Session Management}

Details and discussions about the second subsection.

\section{Conclusion}

The tool is implemented in Python in about 2K lines of code according to the LOC
metric, and has only two Python dependencies so far, at the cost of the
dependency on the operating system intefaces for which we choose POSIX as a
wide-spread openly available standard.

\section*{References}
\bibliographystyle{plain} % Set the bibliography style
\bibliography{paper} % Include the .bib file named 'paper'

\end{document}

